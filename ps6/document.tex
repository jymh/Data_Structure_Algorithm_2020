\documentclass[]{article}
\usepackage{algorithm}
\usepackage{algpseudocode}
\usepackage{amsmath}
\usepackage{forest}

\title{Solution for Problem Set 6}
\author{Mianzhi Pan, 181240045}
\begin{document}
	\maketitle
	\section{Problem 1}
	First randomly choose a bucket from the $m$ buckets.\ Suppose the length of the chosen bucket is $k$,\ we use $RANDOM(1, L)$ and suppose the return value is $a$.\ If $a \leq k$,\ return the $a^{th}$ element in that bucket.\ Otherwise,\ we continue this process until $a \leq k$.\ By this way,\ each element is chosen with probability $\frac{1}{mK}$.
	
	The probability that we succeed in chosing an element in a particular bucket is $\frac{k}{L}$,\ so the expected chosing times are $\frac{L}{k}$.\ Together with $a$ times for retriving the element,\ total time is $O(a + \frac{L}{k}) = O(L\cdot (a/L + 1/k))$,\ so the excepted time is $O(L\cdot (1 + 1/\alpha))$(excepted value of $k$ is $\alpha$ and $a/L \leq 1$).
	
	\section{Problem 2}
	\textbf{(a)}\ Suppose string $x$ of length $l+1$:\ $x_lx_{l-1}\cdots x_0$,\ $x$ has key value $x_l\times (m+1)^l + \cdots + x_0\times (m+1)^0$.\ We have
	\begin{equation*}
	\begin{aligned}
	h(x) &= x\ mod\ m\\
	& = ((x_l\times (m+1)^l)\ mod\ m + \cdots + (x_0\times (m+1)^0)\ mod\ m)\ mod\ m\\
	& = (x_l\ mod\ m + \cdots + x_0\ mod\ m)\ mod\ m
	\end{aligned}
	\end{equation*}
	We can find the hash value of a string is determined by all its characters but is independent with the order of them.\ Hence,\ $x$ and $y$ hash to the same value.
	
	\textbf{(b)}\ Linear probing,\ $h(k,i) = (h'(k) + i)\ mod\ 11$,\ let $h'(k) = k$
	
	\begin{tabular}{|c|}
		\hline
		22\\
		\hline
		88\\
		\hline
		\\
		\hline
		\\
		\hline
		4\\
		\hline
		15\\
		\hline
		28\\
		\hline
		17\\
		\hline
		59\\
		\hline
		31\\
		\hline
		10\\
		\hline
	\end{tabular}

	Quadratic probing:\ $h(k,i) = (k + i + 3i^2)\ mod\ 11$
	
	\begin{tabular}{|c|}
		\hline
		22\\
		\hline
		\\
		\hline
		88\\
		\hline
		17\\
		\hline
		4\\
		\hline
		\\
		\hline
		28\\
		\hline
		59\\
		\hline
		15\\
		\hline
		31\\
		\hline
		10\\
		\hline
	\end{tabular}

	Double hashing:\ $h(k,i) = h_1(k) + ih_2(k)\ mod\ 11$
	
	\begin{tabular}{|c|}
		\hline
		22\\
		\hline
		\\
		\hline
		59\\
		\hline
		17\\
		\hline
		4\\
		\hline
		15\\
		\hline
		28\\
		\hline
		88\\
		\hline
		\\
		\hline
		31\\
		\hline
		10\\
		\hline
	\end{tabular}

	\section{Problem 3}
	Let $B_i = \{w | h(w) = i\}$,\ which respesents the $i^{th}$ bucket in $B$.\ We have
	\begin{equation*}
		\begin{aligned}
		Pr[h(k) = h(l)] &= \frac{\sum_{i=1}^{|B|}C_{|B_i|}^2}{C_{|U|}^2}\\
		& = \frac{\sum_{i=1}^{|B|}|B_i|(|B_i| - 1)}{|U|(|U|-1)}\\
		& = \frac{\sum_{i=1}^{|B|}(|B_i|^2 - |B_i|)}{|U|(|U|-1)}\\
		& = \frac{\sum_{i=1}^{|B|}|B_i|^2 - |U|}{|U|(|U|-1)}\\
		& = \frac{\sum_{i=1}^{|B|}|B_i|^2}{|U|(|U|-1)} - \frac{1}{|U|-1}\\
		& = \frac{\frac{1}{|B|}(\sum_{i=1}^{|B|}|B_i|)^2}{|U|(|U|-1)} - \frac{1}{|U|-1}\\
		& = \frac{|U|}{|B|(|U|-1)} - \frac{1}{|U|-1}\\
		& \geq \frac{|U|-1}{|U|}(\frac{|U|}{|B|(|U|-1)} - \frac{1}{|U|-1})\\
		& = \frac{1}{|B|} - \frac{1}{|U|}
		\end{aligned}
	\end{equation*}
	Notice $Pr[h(k) = h(l)] \leq \epsilon$,\ hence $\epsilon \geq \frac{1}{|B|} - \frac{1}{|U|}$
	
    \section{Problem 4}
    Use \textbf{CircularArray} to implement this D.S.\ The $INSERT(S,x)$ operation is the same as that in class.\ When we do $DELLARGEHALF(S)$,\ we first use $QUICKSELECT$ to select the median,\ then go through all elements and copy those no larger than the median to another half-sized array.
    
    The amortized cost of $INSERT(S,x)$ is $O(1)$ apparently.\ Notice the real cost $c_i$ of $DELLARGEHALF(S)$ is $\Theta(|S|)$,\ i.e.\ $c_i = p|S|$.\ Suppose the potential function is linear to $|S|$,\ i.e.\ $\Phi(D_i) = q|S|$,\ then $\Phi(D_i) - \Phi(D_{i-1}) \leq -\frac{q}{2}|S|$.\ Therefore $\hat{c_i} = (p-\frac{q}{2})|S|$,\ we can always define a potential function such that $\hat{c_i} = O(1)$.\ Hence the amortized cost of $DELLARGEHALF(S)$ is alse $O(1)$.\ The $m$ operations can run in $O(m)$ time in total.
    
    When we want to output the elements,\ we just need to do $REMOVE$ like that in class and output them.  
    
    \section{Problem 5}
    We want to transform this problem to that in which the counter begins at a number with $0$ $1$s.\ First,\ we assume $n \geq cb$.\ In order to transform the $b$ $1$s to $0$s,\ we need to do $b$ times $1\to 0$ operations.\ Average them to $n$ operations,\ each opeartion cost $\frac{b}{n}\leq \frac{1}{c}$ time.\ Together with the origin problem,\ the amortized cost of $INC$ in this problem is no more than $2+\frac{1}{c}$,\ so the total cost is $n(2+\frac{1}{c}) = O(n)$.
\end{document}